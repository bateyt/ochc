\documentclass[]{article}
\usepackage[T1]{fontenc}
\usepackage{lmodern}
\usepackage{amssymb,amsmath}
\usepackage{ifxetex,ifluatex}
\usepackage{fixltx2e} % provides \textsubscript
% use upquote if available, for straight quotes in verbatim environments
\IfFileExists{upquote.sty}{\usepackage{upquote}}{}
\ifnum 0\ifxetex 1\fi\ifluatex 1\fi=0 % if pdftex
  \usepackage[utf8]{inputenc}
\else % if luatex or xelatex
  \ifxetex
    \usepackage{mathspec}
    \usepackage{xltxtra,xunicode}
  \else
    \usepackage{fontspec}
  \fi
  \defaultfontfeatures{Mapping=tex-text,Scale=MatchLowercase}
  \newcommand{\euro}{€}
\fi
% use microtype if available
\IfFileExists{microtype.sty}{\usepackage{microtype}}{}
\usepackage[margin=1in]{geometry}
\ifxetex
  \usepackage[setpagesize=false, % page size defined by xetex
              unicode=false, % unicode breaks when used with xetex
              xetex]{hyperref}
\else
  \usepackage[unicode=true]{hyperref}
\fi
\hypersetup{breaklinks=true,
            bookmarks=true,
            pdfauthor={Oregon Commission on Historic Cemeteries},
            pdftitle={Filming and Photography in Historic Cemeteries},
            colorlinks=true,
            citecolor=blue,
            urlcolor=blue,
            linkcolor=magenta,
            pdfborder={0 0 0}}
\urlstyle{same}  % don't use monospace font for urls
\setlength{\parindent}{0pt}
\setlength{\parskip}{6pt plus 2pt minus 1pt}
\setlength{\emergencystretch}{3em}  % prevent overfull lines
\setcounter{secnumdepth}{0}

%%% Change title format to be more compact
\usepackage{titling}
\setlength{\droptitle}{-2em}
  \title{Filming and Photography in Historic Cemeteries}
  \pretitle{\vspace{\droptitle}\centering\huge}
  \posttitle{\par}
  \author{Oregon Commission on Historic Cemeteries}
  \preauthor{\centering\large\emph}
  \postauthor{\par}
  \date{}
  \predate{}\postdate{}




\begin{document}

\maketitle


{
\hypersetup{linkcolor=black}
\setcounter{tocdepth}{4}
\tableofcontents
}
\subsection{Overview}\label{overview}

The Oregon Commission on Historic Cemeteries (OCHC) has created a number
of position papers to convey its opinion of best practices on various
topics related to historic cemeteries. The following addresses filming
and photography in historic cemeteries and may apply to all cemeteries,
more generally.

\subsection{Cemetery Managers, Managing Agencies, and
Preservationists}\label{cemetery-managers-managing-agencies-and-preservationists}

As the result of changing attitudes and the rapid expansion of
electronic and social media, unauthorized filming and photography in
cemeteries has become a growing issue. More often than not, this occurs
without the knowledge or permission of the owner/caretaker of the
gravesite, or the individual/organization responsible for managing the
cemetery. Regardless of intent, posting of such items can be quite
upsetting to descendants.

Unfortunately, the legal system has failed to formally address this
issue. The best strategy is to have regulations and a policy in place to
addressing such issues. This information, along with the other rules for
your cemetery, need to be prominently posted and available for all
visitors to your cemetery.

If your cemetery does not already have a policy in place regarding this
subject, you may want to consider doing so. For example, here is the
language for
\href{http://www.oregonmetro.gov/historic-cemeteries/visiting-cemeteries}{Portland
Metro cemeteries}:

\begin{quote}
Use discretion when filming or photographing the landscape and graves.
Commercial filming, photography or videography requires a special use
permit. You are not allowed to\ldots{}take videos or photos of people
visiting a gravesite or at a gravesite service without their permission.
\end{quote}

A number of other examples of language for guidelines regarding filming
and photography in cemeteries has been compiled in a
\href{http://www.legalgenealogist.com/blog/2012/10/22/cemetery-photos-permission-required/}{blog
post by Judy Russell}. These may assist you in getting a conversation
started with your cemetery group or organization as to the possible need
for putting a policy in place for your cemetery.

Should unauthorized pictures be found on the Internet or other
electronic media, a formal, written request to the person or
organization responsible, asking that they be removed and what action
will be taken if they are not, will in most cases help resolve the
matter. Get the support of your Cemetery Commission, Mayor and City
Council, Police Department and if possible, members of the family or
families whose gravesites may have been photographed. Be sure to copy
them all in on the letter as it will make your letter that much more
credible, stronger and hopefully more meaningful.

\subsection{Filmmakers and
Photographers}\label{filmmakers-and-photographers}

Do you need permission to film or take photographs in a cemetery? As
Judy Russell points out, the answer to this question is within the
purview of property rights.

\begin{quote}
Now it may seem strange to think of cemeteries as property, particularly
when they're owned by a governmental entity, but any landowner---public
or private---has certain rights to control what happens on that land.
Even when the land is publicly owned and dedicated to a public purpose,
such as a park, the landowner is absolutely entitled to impose time,
place and manner restrictions as to what can and can't be done on the
land (Russell 2012).
\end{quote}

If you are considering filming or taking photographs within a historical
cemetery (or any cemetery, for that matter), the predicament is that
there are no formal recommendations on this issue. At the same time, a
simple, guiding principle is to treat the cemetery and those buried in
it with respect and dignity. For general guidance, one may consider the
following two sets of professional standards:

\begin{itemize}
\item
  \href{https://nppa.org/code_of_ethics}{Code of Ethics of the National
  Association of Press Photographers}
\item
  Image-related \href{https://diigo.com/05ns77}{News Values and
  Principles of the Associated Press}
\end{itemize}

\subsection{Summary Statement}\label{summary-statement}

The OCHC supports responsible, respectful, and sensitive filming and
photography in our historic cemeteries. We encourage those considering
these activities to check with the cemetery owners regarding applicable
regulations \emph{before} proceeding.

\subsection{Information and Additonal
Resources}\label{information-and-additonal-resources}

For information, advising, and additional resources, please contact
\href{mailto:Kuri.Gill@oregon.gov}{Kuri Gill}, Grants and Outreach
Coordinator for Oregon Heritage and Program Coordinator for the
\href{http://www.oregon.gov/oprd/HCD/OCHC/Pages/index.aspx}{OCHC}.

\subsection{Bibliography}\label{bibliography}

``Code of Ethics.'' National Association of Press Photographers.
\url{https://nppa.org/code_of_ethics}.

``News Values and Principles.'' Associated Press.
\url{http://www.ap.org/company/news-values}.

Russell, Judy G. 2012. ``Cemetery Photos: permission Required?''
\url{http://www.legalgenealogist.com/blog/2012/10/22/cemetery-photos-permission-required/}.

\end{document}
