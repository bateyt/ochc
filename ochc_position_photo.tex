\documentclass[]{article}
\usepackage[T1]{fontenc}
\usepackage{lmodern}
\usepackage{amssymb,amsmath}
\usepackage{ifxetex,ifluatex}
\usepackage{fixltx2e} % provides \textsubscript
% use upquote if available, for straight quotes in verbatim environments
\IfFileExists{upquote.sty}{\usepackage{upquote}}{}
\ifnum 0\ifxetex 1\fi\ifluatex 1\fi=0 % if pdftex
  \usepackage[utf8]{inputenc}
\else % if luatex or xelatex
  \ifxetex
    \usepackage{mathspec}
    \usepackage{xltxtra,xunicode}
  \else
    \usepackage{fontspec}
  \fi
  \defaultfontfeatures{Mapping=tex-text,Scale=MatchLowercase}
  \newcommand{\euro}{€}
\fi
% use microtype if available
\IfFileExists{microtype.sty}{\usepackage{microtype}}{}
\usepackage[margin=1in]{geometry}
\ifxetex
  \usepackage[setpagesize=false, % page size defined by xetex
              unicode=false, % unicode breaks when used with xetex
              xetex]{hyperref}
\else
  \usepackage[unicode=true]{hyperref}
\fi
\hypersetup{breaklinks=true,
            bookmarks=true,
            pdfauthor={Oregon Commission on Historic Cemeteries},
            pdftitle={Filming and Photography in Historic Cemeteries},
            colorlinks=true,
            citecolor=blue,
            urlcolor=blue,
            linkcolor=magenta,
            pdfborder={0 0 0}}
\urlstyle{same}  % don't use monospace font for urls
\setlength{\parindent}{0pt}
\setlength{\parskip}{6pt plus 2pt minus 1pt}
\setlength{\emergencystretch}{3em}  % prevent overfull lines
\setcounter{secnumdepth}{0}

%%% Change title format to be more compact
\usepackage{titling}
\setlength{\droptitle}{-2em}
  \title{Filming and Photography in Historic Cemeteries}
  \pretitle{\vspace{\droptitle}\centering\huge}
  \posttitle{\par}
  \author{Oregon Commission on Historic Cemeteries}
  \preauthor{\centering\large\emph}
  \postauthor{\par}
  \date{}
  \predate{}\postdate{}




\begin{document}

\maketitle


{
\hypersetup{linkcolor=black}
\setcounter{tocdepth}{4}
\tableofcontents
}
\subsection{Overview}\label{overview}

The Oregon Commission on Historic Cemeteries (OCHC) has created a number
of position papers to convey their opinion of best practices on various
topics related to historic cemeteries. The following addresses filming
and photography in historic cemeteries and may apply to all cemeteries,
more generally.

\subsection{Cemetery Managers, Managing Agencies, and
Preservationists}\label{cemetery-managers-managing-agencies-and-preservationists}

If your cemetery has not had any problems or concerns with unauthorized
filming or photography, consider yourself fortunate, as this is becoming
more of a problem as attitudes change and social and electronic media
continues to grow and expand.

With our ever changing world and the constant introduction of new forms
of instant media and communication gadgets, cemeteries may no longer be
the final resting places they once were. They and the gravesites
contained within their grounds are finding their way on to websites,
face book and other social media sites. In the majority of these cases,
it is without the knowledge or permission of the owner or caretaker of
the gravesite, or the cemetery owner.

While the intentions of some of those taking photos may be good, others
are not and can be quite upsetting to the families of loved ones who
discover the pictures on line, sometimes for sale. The fact remains that
these gravesites are private property and owned by someone. Cemeteries,
whether public or private, have certain rights to control what happens
on that land. Even when the land is publicly owned and dedicated to a
public purpose, such as a park, the landowner is absolutely entitled to
impose time, place and manner restrictions as to what can and can’t be
done on the land.

While laws have failed to keep up with dealing with problems resulting
from modern day social media practices and it may be difficult to
enforce filming and photography in our cemeteries, it all starts with
your cemetery having rules, regulations and a policy in place addressing
the subject. This, along with the other rules and regulations for your
cemetery, need to be posted and available for all to see and be made
aware of.

If your cemetery does not already have a policy in place regarding this
subject, you may want to consider doing so. Some examples of policies
already in place at other cemeteries may assist you in developing a
policy for your cemetery. They are as follows:

\href{http://www.cityoflafayette.com/DocumentCenter/Home/View/443}{Coal
Creek Memorial Cemetery}---Louisville, Colorado

\begin{quote}
Filming and photography for use in a movie, book, newspaper, magazine,
television news, paranormal research, Internet or other electronic media
are not permitted on Cemetery grounds.
\end{quote}

\href{http://www.oregonmetro.gov/historic-cemeteries/visiting-cemeteries}{Metro
Cemeteries}---Portland, Oregon

\begin{quote}
Use discretion when filming or photographing the landscape and graves.
Commercial filming, photography or videography requires a special use
permit. You are not allowed to\ldots{}take videos or photos of people
visiting a gravesite or at a gravesite service without their permission.
\end{quote}

Woodlawn Cemetery – Everett, Massachusetts Photographing funerals is
prohibited. Commercial photographs and movie making are not permitted
except by special permission from the cemetery.

Green-Wood Cemetery – Brooklyn, New York Photography is permitted.
Please ask for a copy of our Photography Policy at our main entrance.
Professional photography, including use of light, stands or other
equipment, as well as publishing of any photographs taken at Green-Wood,
requires written consent of The Green-Wood Cemetery. Filming and
videotaping are strictly prohibited without advance permission and
require written consent of The Green-Wood Cemetery.

Glenwood Cemetery- Houston, Texas Photography for private (not
commercial) use is permitted so long as it does not interfere with the
quiet enjoyment of the cemetery by other visitors. Photography in
available light is preferred, although flash cameras may be used.
External light sources not integral to the camera may not be used.
Photography of burials is permitted only with the express permission of
the person authorizing the burial, and such permission should be made
know to the Glenwood office in advance of the burial. Photography for
commercial use is prohibited, except with the written permission of the
Executive Director. Requests should be submitted to the Glenwood office.

Brookwood Cemetery – Woking, Surrey, England Interestingly the cemetery
has had restrictions on photography since its opening in 1854. We do not
permit any photographs to be taken in the cemetery unless you have a
photographic permit issued by the Cemetery Office. These permits
prohibit the posting of photographs on the internet. If you wish to do
this you will need to apply for further permission from the Cemetery
Office. All requests for photography must be made in writing to the
Cemetery Office and permits will be issued at our discretion. There is a
suggested donation of 10 pounds, payable to the Brookwood Cemetery
Restoration Fund.

These are just a few examples that may assist you in getting a
conversation started with your cemetery group or organization as to the
possible need for putting a policy in place for your cemetery.

In the meantime, should unauthorized pictures be found on the Internet
or other electronic media, a letter to the person or organization
responsible, asking that they be removed and what action will be taken
if they are not, will in most cases help resolve the matter. Get the
support of your Cemetery Commission, Mayor and City Council, Police
Department and if possible, members of the family or families whose
gravesites may have been photographed. Be sure to copy them all in on
the letter as it will make your letter that much more credible, stronger
and hopefully more meaningful.

\subsection{Filmmakers and
Photographers}\label{filmmakers-and-photographers}

\subsection{Summary}\label{summary}

The OCHC supports responsible, respectful and sensitive preservation and
educational photography and filming in our historic cemeteries, and
encourages those wanting to do so, to check with the cemetery owners
with regard to the rules and regulations before proceeding.

\subsection{Information and Additonal
Resources}\label{information-and-additonal-resources}

For information, advising, and additional resources, please contact
\href{mailto:Kuri.Gill@oregon.gov}{Kuri Gill}, Grants and Outreach
Coordinator for Oregon Heritage and Program Coordinator for the OCHC, or
one of the OCHC Commissioners.

\subsection{References}\label{references}

\end{document}
